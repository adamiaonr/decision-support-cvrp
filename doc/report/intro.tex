\section{Introduction}
\label{sec:intro}

The Vehicle Routing Problem (VRP) plays an important part in the areas of 
Operations Research related with physical distribution and logistics. The VRP is 
associated with real-world applications concerned with the distribution of 
supplies between depots and demand points or stations, using vehicles as transport for 
those same supplies. Among the variations of the VRP is 
the Capacitated VRP (CVRP), which will be the object of study for the remainder 
of this paper.

\subsection{Definition}
\label{subsec:definition}

Consider $N$ demand points or stations $P_i$, with $i = 1, 2, ..., N$, in a 
given region, each requesting a 
quantity\slash demand $\sigma_i$ of supplies to be 
delivered to it. The supplies are kept at a depot $D$, which also keeps a 
fleet of $V$ transporting vehicles. Vehicles have equal maximum carrying capacities 
$C$. Each vehicle must start and finish its route at the depot $D$. The 
problem is to get a group of delivery routes from the depot $D$ to the $N$ 
stations, so that the total distance covered by the whole fleet of 
vehicles is minimized. One assumes the demands $\sigma_i$ are less than the maximum 
capacity of each vehicle, and the whole quantity $\sigma_i$ at a station $P_i$ 
must be satisfied by a single vehicle.\vertbreak

Notice that the VRP reduces to a Traveling Salesman Problem (TSP) when the 
vehicle capacity constraint is relaxed: if the objective is to minimize 
total distance, it is a 1-TSP and a n-TSP if a given number of vehicles $V$ is 
specified. Following this reasoning, the TSP can be looked at as a special case 
of the VRP.

\subsection{Constructive Heuristics for the CVRP}
\label{subsec:constructive}

The Clarke Wright 
Savings (CWS) algorithm, as presented in Section 5.2.1 of~\cite{Toth2002}, 
is a widely known approach to tackle CVRP. The 
key idea behind the CWS algorithm is the concept of `savings'. In order to 
better understand this concept --- and keeping the definition of the 
CVRP in mind --- let us consider a depot $D$ and $N$ stations $P_i$, with $i = 1, 2, ..., N$. Now consider an 
initial solution to the CVRP\footnote{In this context, a solution $S$ to the CVRP is represented as a group of $V$ routes, each one serviced by one of the $V$ available vehicles, which pass through all demand points $P$ when taken together. More details on Section~\ref{subsubsec:sol-rep}.} consisting of $N$ routes (serviced by $N$ vehicles), 
dispatching one vehicle to each one of the $N$ stations. For each one of 
the routes to a station $P_i$, with $i = 1, 2, ..., N$, the length is 
$2 \times d(D,P_i)$\footnote{We use the total distance between stations, of 
a route or of a complete solution $S$, $d(.)$ as the evaluation function for the 
CVRP. Check Section~\ref{subsubsec:eval-fun} for details.}, i.e $d(D,P_i) + d(P_i,D)$. For all the routes, the overall cost (i.e. 
length) becomes $2\sum_{i=1}^{N} d(D,P_i)$.\vertbreak

If one now dispatches a single vehicle to satisfy two stations $P_i$ and $P_j$, on a 
single route, the total distance traveled is reduced by the amount $s_{ij}$, i.e. 
the `saving' one gets by servicing two stations at once vs. servicing only 
one station:

\begin{equation*}
\begin{split}
s_{ij}  &= 2d(D,P_i) + 2d(D,P_j) - (d(D,P_i) + d(P_i,P_j) + d(P_j,D))\\
        &= d(D,P_i) + d(P_j,D) - d(P_i,P_j)
\end{split}
\end{equation*}\vertbreak

The larger the `saving', the better it is to combine the link $(P_i,P_j)$ in a single 
CVRP route. The Clarke and Wright Savings algorithm follows this concept. The 
algorithm is composed by the following steps, as described in~\cite{Toth2002}, 
section 5.2.1:

\begin{itemize}

    \item \textbf{STEP 1:} Compute the savings $s_{ij}$ for each combination of 
            stations $P_i, P_j$, with $i,j = 1, ..., N \wedge i \neq j$. Create $N$ vehicle 
            routes $(D,P_i,D)$ for $i = 1, ..., N$. Order the savings $s_{ij}$ in 
            a descending order of magnitude.
    \item \textbf{STEP 2 (Parallel Version):} Start from the top of the 
            savings list. Given a saving $s_{ij}$, determine whether there 
            exist two routes, one containing link $(D,P_j)$ and the other 
            containing link $(P_i, D)$, that can feasibly be merged. If so, 
            combine these two routes by deleting $(D,P_j)$ and $(P_i,D)$ and 
            introducing $(P_i,P_j)$.
    \item \textbf{STEP 2 (Sequential Version):} Consider in turn each route 
            $(D,P_i,...,P_j,D)$. Determine the first saving $s_{ki}$ or $s_{jl}$ 
            that can 
            feasibly be used to merge the current route with another route 
            containing link $(P_k, D)$ or containing link $(D, P_l)$. 
            Implement the merge and repeat this operation to the current route. 
            If no feasible merge exists, consider the next route and reapply 
            the same operations. Stop when no route merge is feasible.
\end{itemize}\vertbreak

By analyzing the description of the CWS algorithm given above, one can classify 
it as a clearly constructive heuristic, due to its principle of building 
`complex' routes by merging more simple routes --- initially in the form 
$(D,P_i,D)$ --- via links $(P_i,P_j)$, associated with a special quantity designated 
by `savings'.

\subsection{Metaheuristics for the CVRP}
\label{subsec:metaheuristics}

Metaheuristics are high-level strategies designed to explore the solution space 
in search of near-optimal solutions, which normally combine different 
lower-level heuristics such as the CWS algorithm described in 
Section~\ref{subsec:constructive}. In contrast to traditional heuristics, 
metaheuristics often accept solutions which are worse than a current solution 
$S_c$ during a run, allowing for a more extensive search and the escape from 
local optima.\vertbreak

In this paper we present a custom implementation of two metaheuristics, adapted 
for the CVRP: Simulated Annealing and Genetic Algorithms.\vertbreak

\subsubsection{Simulated Annealing}
\label{subsubsec:sim-anneal}

Simulated Annealing (SA) is an example of a probabilistic metaheuristic, which 
has been applied to the CVRP by several authors~\cite{Toth2002}. SA is based on 
the following rule: given a neighbor $S_n$ of a current solution $S_c$, if 
better that $S_c$, accept it as the new $S_c$; otherwise, either 
probabilistically accept this new weaker $S_n$ or continue to search in 
the current neighborhood. The probability of accepting weaker $S_n$ depends 
on a parameter $T$ --- a temperature --- which decreases, i.e. `cools down', as 
the algorithm advances, making such acceptances less likely to occur.\vertbreak

The version of the Simulated Annealing (SA) algorithm implemented in this work 
does not significantly differ from its usual form, as presented in Figure 5.3 
in~\cite{Michalewicz2004}\footnote{Note that the SA structure shown in Figure 
5.3 in~\cite{Michalewicz2004} assumes a maximization problem while we assume 
minimization.} (values of $L$ and $M$ are defined a priori):\vertbreak

\begin{algorithmic}[1]

\State $k \leftarrow 0$
\State $l \leftarrow 0$
\State initialize $T$
\State generate a solution $S_c$ via the CWS algorithm
\State $S_b \leftarrow S_c$
\State $S_{\text{unchanged}} \leftarrow 0$

\Repeat
\Repeat

\State generate new $S_n$, in the neighborhood of $S_c$

\If {$d(S_n) < d(S_c)$}

    \State $S_c \leftarrow S_n$
    \State $S_{\text{unchanged}} \leftarrow 0$

    \If {$d(S_n) < d(S_b)$}

        \State $S_b \leftarrow S_n$

    \EndIf

\ElsIf {random[0,1) $< e^{\frac{d(S_c) - d(S_n)}{T}}$}

    \State $S_c \leftarrow S_n$
    \State $S_{\text{unchanged}} \leftarrow 0$

\Else

    \State $S_{\text{unchanged}} \leftarrow S_{\text{unchanged}} + 1$

\EndIf

\State $l \leftarrow l + 1$

\Until{$l < L$}

\State $T \leftarrow T \times (1 - \alpha)$
\State $k \leftarrow k + 1$

\Until{$S_{\text{unchanged}} < M$}

\end{algorithmic}\vertbreak

\subsubsection{Genetic Algorithms}
\label{subsubsec:gen-al}

Genetic Algorithms (GA) try to mimic the process of evolution verified in nature 
by modeling the natural processes of competition, reproduction and mutation, over 
a population of individuals $W$. In this case, the population $W$, at a given 
time (or generation) $t$ is composed by $|W|$\footnote{We will henceforth use the notation $|W|$ to represent the size of a population of solutions $W$.} different solutions $S_n$ to the 
CVRP. This is one of the main differences of GAs when compared to other 
metaheuristics, the maintenance of $|W|$ solutions at a given iteration of the 
procedure, instead of just one as in SA.\vertbreak

The evolutionary approach implemented here follows a typical sequence of steps, 
as presented in Figure 6.5 in~\cite{Michalewicz2004}:\vertbreak

\begin{algorithmic}[1]

\State $t \leftarrow 0$
\State initialize $W(t)$
\State $S_b \leftarrow$ best of $W(t)$

\While{$t < G$}

\State $t \leftarrow t + 1$

\State select $W'(t)$ from $W(t - 1)$

\State generate offspring set $O(t)$ from $W'(t)$

\State apply mutations over $O(t)$, 
\Statex[2] with probability $M \in [0.0, 1.0]$

\State join $W'(t)$ and $O'(t)$ into a new population $W(t)$

\If {best of $W(t) < d(S_b)$}

    \State $S_b \leftarrow$ best of $W(t)$
    \State save $t$

\EndIf

\EndWhile

\end{algorithmic}\vertbreak

In the listing shown above, $t$ is a generation index; $W(t)$ represents the 
population of solutions (of size $|W|$, given as a parameter to the algorithm) at 
generation $t$; $S_b$ corresponds to the 
best solution found by the algorithm as it executes; $G$ represents the number of generations 
to be created (also the stopping condition of the algorithm), defined a priori; 
$W'(t)$ represents the set of solutions after the application of a Random 
Tournament Selection strategy (explained in detail in Section~\ref{subsec:random-tournament}); $O(t)$ 
is the set of `offspring' solutions generated by the elements of $W'(t)$, using 
a variant of Partially-Mapped Crossover (PMX) (see Section~\ref{subsec:random-tournament}); $O'(t)$ is 
resulting set of solutions after applying random mutations over the set $O(t)$, 
with a probability $M \in [0.0, 1.0]$, also defined a priori (see Section~\ref{subsec:mutations}).


